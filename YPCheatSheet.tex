\documentclass[9pt,oneside]{amsart}
\usepackage{amsmath}
\usepackage{amssymb}
\usepackage[a4paper,width=170mm,top=18mm,bottom=22mm,includeheadfoot]{geometry}
\usepackage{booktabs}
\usepackage[usenames,dvipsnames]{xcolor}
\usepackage{longtable}
\usepackage{array}

\definecolor{lightyellow}{rgb}{1,0.98,0.9}

\begin{document}

\pagecolor{lightyellow}

\section{Conventions}

\center

\newcolumntype{L}[1]{>{\raggedright\let\newline\\\arraybackslash\hspace{0pt}}p{#1}}
\begin{tabular}{L{0.25\linewidth}L{0.25\linewidth}L{0.3\linewidth}}
\toprule
\textbf{Item} & \textbf{Convention} & \textbf{Examples} \\
\midrule
Top level structures & Lower case bold Greek & $\boldsymbol{\sigma}$, the world state\newline $\boldsymbol{\mu}$, the machine state. \\
\midrule
Functions on highly structured values & Upper case Greek & $\Upsilon$, the Ethereum state transition function. \\
\midrule
Most functions & Upper case letters, possibly subscripted	& $C$, the general cost function\newline $C_\text{\tiny SSTORE}$, the cost function for the {\tiny SSTORE} operation. \\
\midrule
Specialised functions & Typewriter & $\texttt{KEC}$, the Keccak-256 hash\newline $\texttt{KEC512}$, the Keccak-512 hash function. \\
\midrule
Tuple & Upper case letter & $T$, a transaction. \\
\midrule
Component of a Tuple & Subscripted upper-case letter.\newline A capital subscript refers to a component that is a tuple. & $T_n$, the transaction nonce\newline $I_H$, The header of the current block (a tuple). \\
\midrule
Scalars, fixed size byte sequences/arrays & Usually a lower-case letter Sometimes Greek & $n$, a transaction's nonce\newline $\delta$, the number of stack items required. \\
\midrule
Arbitrary length sequences & Bold lower-case & $\mathbf{o}$, output data of message call. \\
\midrule
Sets & Double struck capitals & $\mathbb{P}_{256}$, positive integers less than $2^{256}$\newline $\mathbb{B}_{32}$, byte sequences of length $32$. \\
\midrule
Components or subsequences of sequences & Square brackets & $\boldsymbol{\mu}_\mathbf{s}[0]$, the first item on the stack\newline $\boldsymbol{\mu}_\mathbf{m}[0..31]$ the first $32$ items in memory. \\
\midrule
Modified (and utilisable) value & Prime mark & $g'$ gas remaining. \\
\midrule
Intermediate values & Asterisk superscripts & $g^*$ gas to be refunded\newline $g^{**}$ available gas remaining after code execution.\\
\midrule
Element-wise transformations & Asterisk superscript on a function & $f^*\big((x_0, x_1, ...) \big) \equiv \big( f(x_0), f(x_1), ... \big)$ for any function $f$. \\
\bottomrule
\end{tabular}

\endcenter

\vspace{7pt}
\section{Symbols}

\begin{longtable}{p{0.10\linewidth}p{0.85\linewidth}}
\toprule
Name & Description \\
\midrule
\endhead
\multicolumn{2}{l}{\textbf{High level constructs}} \\*[5pt]
$\boldsymbol{\sigma}$ & The world-state, comprising all accounts' nonces, balances, storage and code. \\
$\boldsymbol{\sigma}_t$ & World-state at time $t$. \\
$\boldsymbol{\mu}$ & Machine-state tuple, $(g, pc, \mathbf{m}, i, \mathbf{s})$, which are gas, program counter, memory, memory size, stack. \\
$T$ & An Ethereum transaction \\
$T_0$, $T_1$, ... & Individual transactions within a block \\
$B$ & A block: $B \equiv (..., (T_0, T_1, ...) )$ \\
$\Upsilon$ & The Ethereum state transition function: $\boldsymbol{\sigma}_{t+1} \equiv \Upsilon(\boldsymbol{\sigma}_t, T)$ \\
$\Omega$ & The block-finalisation state transition function (pays out the mining reward). \\
$\Pi$ & The block-level state-accumulation function: $\Pi(\boldsymbol{\sigma}, B) \equiv \Omega(B, \Upsilon(\Upsilon(\boldsymbol{\sigma}, T_0), T_1) ...)$ \\

\vspace{5pt} \\
\midrule
\multicolumn{2}{l}{\textbf{World state}} \\*[5pt]
$\boldsymbol{\sigma}[a]$ & The account state of account $a$, being a tuple of (nonce, balance, storageRoot, codeHash). \\
$\boldsymbol{\sigma}[a]_n$ & The nonce of account $a$. \\
$\boldsymbol{\sigma}[a]_b$ & The balance of account $a$. \\
$\boldsymbol{\sigma}[a]_s$ & A 256-bit hash of the root node of a Merkle Patricia tree that encodes the storage contents of account $a$. Note that $\texttt{\small TRIE}\big(L_I^*(\boldsymbol{\sigma}[a]_\mathbf{s})\big) \equiv \boldsymbol{\sigma}[a]_s$ \\
$\boldsymbol{\sigma}[a]_c$ & The hash of the EVM code of account $a$. Equal to $\texttt{\small KEC}(\mathbf{b})$ where $\mathbf{b}$ is the account's code.\\

\vspace{5pt} \\
\midrule
\multicolumn{2}{l}{\textbf{Machine state}} \\*[5pt]
$\boldsymbol{\mu}_g$ & The gas available. \\
$\boldsymbol{\mu}_{pc}$ & The program counter. \\
$\boldsymbol{\mu}_\mathbf{m}$ & The memory contents. \\
$\boldsymbol{\mu}_i$ & The number of memory words allocated. \\
$\boldsymbol{\mu}_\mathbf{s}$ & The stack. \\
$\boldsymbol{\mu}_\mathbf{s}[n]$ & Item at stack depth $n$. \\

\vspace{5pt} \\
\midrule
\multicolumn{2}{l}{\textbf{Substate}} \\*[5pt]
$A$ & A Transaction substate during execution: $\equiv (A_\mathbf{s}, A_\mathbf{l}, A_r) \equiv A \equiv (\mathbf{s}, \mathbf{l}, r)$. \\
$A_\mathbf{s}$ & The self-destruct set. \\
$A_\mathbf{l}$ & The log series. \\
$A_r$ & The gas refund balance. Can partially offset execution costs.\\
$A^0$ & The empty substate: $A^0 \equiv (\varnothing, (), 0)$. \\

\vspace{5pt} \\
\midrule
\multicolumn{2}{l}{\textbf{Execution environment}} \\*[5pt]
$I$ & Tuple of the following items provided to the execution environment. \\
$I_a$ & The address of the account which owns the code that is executing. \\
$I_o$ & The sender address of the transaction that originated this execution. \\
$I_p$ & The price of gas in the transaction that originated this execution. \\
$I_\mathbf{d}$ & The byte array that is the input data to this execution; if the execution agent is a transaction, this would be the transaction data. \\
$I_s$ & The address of the account which caused the code to be executing; if the execution agent is a transaction, this would be the transaction sender. \\
$I_v$ & The value, in Wei, passed to this account as part of the same procedure as execution; if the execution agent is a transaction, this would be the transaction value. \\
$I_\mathbf{b}$ & The byte array that is the machine code to be executed. \\
$I_H$ & The block header of the present block. \\
$I_e$ & The depth of the present message-call or contract-creation (i.e. the number of {\small CALL}s or {\small CREATE}s being executed at present). \\
$I_w$ & Flag for permission to make modifications to the state. See EIP-214, STATICCALL \\

\vspace{5pt} \\
\multicolumn{2}{l}{Execution} \\*[5pt]
$\Xi$ & The code execution function $(\boldsymbol{\sigma}', g', A, \mathbf{o}) \equiv \Xi(\boldsymbol{\sigma}, g, I)$. \\
$\mathbf{o}$ & The output data of a message call, $\mathbf{o} \equiv H(\boldsymbol{\mu}, I)$.\newline At contract creation, the contract bytecode to be deployed. \\
$H(\boldsymbol{\mu}, I)$ & The normal halting function, usually the value provided by the RETURN or REVERT opcodes, or empty in the case of STOP. \\
$Z(\boldsymbol{\sigma}, \boldsymbol{\mu}, I)$ & The exceptional halting function. \\
$w$ & The current operation to be executed: $w \equiv I_\mathbf{b}[\boldsymbol{\mu}_{pc}]$ if $\boldsymbol{\mu}_{pc} < \lVert I_\mathbf{b} \rVert$, and \small{STOP} otherwise. \\

\vspace{5pt} \\
\midrule
\multicolumn{2}{l}{\textbf{Blocks}} \\*[5pt]
$B$ & A block: $B \equiv (B_H, B_\mathbf{T}, B_\mathbf{U}).$ \\
$B_H$ & The block's header. \\
$B_\mathbf{T}$ & The block's transactions. \\
$B_\mathbf{U}$ & Headers of ommer/uncle blocks of this block. \\
$B_\mathbf{R}$ & Transaction receipts. \\
$D(H)$ & The difficulty of the block with header $H$. \\
$P(H)$ & The parent block of the block with header $H$. \\
$V(H)$ & The block header validity function. \\

\vspace{5pt} \\
\multicolumn{2}{l}{Block header} \\*[5pt]
$H_p$ & \textbf{parentHash}: The Keccak 256-bit hash of the parent block's header, in its entirety. \\
$H_o$ & \textbf{ommersHash} The Keccak 256-bit hash of the ommers list portion of this block. \\
$H_c$ & \textbf{beneficiary} The 160-bit address to which all fees collected from the successful mining of this block be transferred. \\
$H_r$ & \textbf{stateRoot} The Keccak 256-bit hash of the root node of the state trie, after all transactions are executed and finalisations applied. \\
$H_t$ & \textbf{transactionsRoot} The Keccak 256-bit hash of the root node of the trie structure populated with each transaction in the transactions list portion of the block. \\
$H_e$ & \textbf{receiptsRoot} The Keccak 256-bit hash of the root node of the trie structure populated with the receipts of each transaction in the transactions list portion of the block. \\
$H_b$ & \textbf{logsBloom} The Bloom filter composed from indexable information (logger address and log topics) contained in each log entry from the receipt of each transaction in the transactions list. \\
$H_d$ & \textbf{difficulty} A scalar value corresponding to the difficulty level of this block. \\
$H_i$ & \textbf{number} A scalar value equal to the number of ancestor blocks. The genesis block has a number of zero. \\
$H_l$ & \textbf{gasLimit} A scalar value equal to the current limit of gas expenditure per block. \\
$H_g$ & \textbf{gasUsed} A scalar value equal to the total gas used in transactions in this block. \\
$H_s$ & \textbf{timestamp} A scalar value equal to the reasonable output of Unix's time() at this block's inception. \\
$H_x$ & \textbf{extraData} An arbitrary byte array containing data relevant to this block. This must be 32 bytes or fewer. \\
$H_m$ & \textbf{mixHash} A 256-bit hash which proves combined with the nonce that a sufficient amount of computation has been carried out on this block. \\
$H_n$ & \textbf{nonce} A 64-bit hash which proves combined with the mix-hash that a sufficient amount of computation has been carried out on this block. \\

\vspace{5pt} \\
\midrule
\multicolumn{2}{l}{\textbf{Transactions}} \\*[5pt]
$T_n$ & Transaction nonce. \\
$T_p$ & Gas price for the transaction. \\
$T_g$ & The maximum gas for a transaction. \\
$T_t$ & The ``to'' address for the transaction. \\
$T_v$ & The value to be transferred by the transaction. \\
$T_w$, $T_r$, $T_s$ & The $v$, $r$, $s$ values of the transaction signature. \\
$T_\mathbf{i}$ & EVM-code for account initialisation (i.e. contract deployment). \\
$T_\mathbf{d}$ & Input data of a message call. \\
$S(T)$ & Sender function---recovers the sender address from the transaction: \newline $S(T) \equiv \mathcal{B}_{96..255}\big(\mathtt{KEC}\big( \mathtt{ECDSARECOVER}(h(T), T_w, T_r, T_s) \big) \big).$ \\

\vspace{5pt} \\
\multicolumn{2}{l}{Transaction Receipt} \\*[5pt]
$R$ & A transaction receipt: $R \equiv (R_u, R_b, R_\mathbf{l}, R_z)$ \\
$R_u$ & The cumulative gas used so far in the block. \\
$R_b$ & The bloom filter composed from the information in the transaction logs. \\
$R_\mathbf{l}$ & The log entries created by the transaction, $(O_0, O_1, ...)$. \\
$R_z$ & The status code of the transaction. \\
$O$ & A log entry: $O \equiv (O_a, ({O_\mathbf{t}}_0, {O_\mathbf{t}}_1, ...), O_\mathbf{d})$. \\
$O_a$ & The logger's address. \\
$O_\mathbf{t}$ & A 32-byte log topic. \\
$O_\mathbf{d}$ & The log data for this entry. \\

$\Upsilon^g$ & The total gas used in this transaction. \\
$\Upsilon^\mathbf{l}$ & The logs created by this transaction. \\
$\Upsilon^z$ & The status code of this transaction, $z$. \\

\vspace{5pt} \\
\midrule
\multicolumn{2}{l}{\textbf{Miscellaneous functions}} \\*[5pt]
$\ell(\mathbf{x})$ & The last item in sequence $\mathbf{x}$: $\ell(\mathbf{x}) \equiv \mathbf{x}[\lVert \mathbf{x} \rVert - 1]$ \\
$L(n)$ & The ``all but one 64th'' function: $L(n) \equiv n - \lfloor n / 64 \rfloor$.\\
$L_I\big( (k, v) \big)$ & Representation of key--value pairs in the trie: $L_I\big( (k, v) \big) \equiv \big(\texttt{KEC}(k), \texttt{RLP}(v)\big)$ \\
$L_R$ & TODO \\
$L_S$ & World-state collapse function. TODO: expand. Seems to have a different function in computing the message hash.\\
$L_T$ & TODO \\
$M(s, f, l)$ & Memory expansion function. $s$ is the current top of memory; $f$ is the start of writing; $l$ is the number of bytes to be written. \\
$\mathcal{B}$ & Bit reference function such that $\mathcal{B}_j(\mathbf{x})$ equals the bit of index $j$ (indexed from 0) in the byte array $\mathbf{x}$ \\
$\mathtt{EMPTY}(\boldsymbol{\sigma}, a)$ & An account $a$ is \textit{empty} when it has no code, zero nonce and zero balance, $\boldsymbol{\sigma}[a]_c = \texttt{\small KEC}\big(()\big) \wedge \boldsymbol{\sigma}[a]_n = 0 \wedge \boldsymbol{\sigma}[a]_b = 0$. \\
$\mathtt{DEAD}(\boldsymbol{\sigma}, a)$ & An account $a$ is \textit{dead} when its account state is non-existent or empty: $\varnothing \vee \mathtt{EMPTY}(\boldsymbol{\sigma}, a)$. \\
$\mathtt{TRIE}$ & The root hash of the Merkle Patricia tree constructed from its arguments. \\
$\mathtt{KEC}$ & TODO \\
$\mathtt{RLP}$ & TODO \\
$\mathtt{PoW}$ & TODO \\

\vspace{5pt} \\
\midrule
\multicolumn{2}{l}{\textbf{Operators and symbols}} \\*[5pt]
$\lVert ... \rVert$, $| ... |$ & Length of a sequence. These seem to be used interchangeably, but I may have missed something. \\
$\wedge$ & Logical ``And''. \\
$\vee$ & Logical ``Or''. \\
$\varnothing$ & The empty set. \\
$\cdot$ & Concatenation, $(a, b, c, d) \cdot e \equiv (a, b, c, d, e)$, or scalar multiplication depending on context. \\

\vspace{5pt} \\
\midrule
\multicolumn{2}{l}{\textbf{Todo}} \\*[5pt]
$\mathbb{B}$ & The set of all sequences of bytes. \\
$\mathbb{B}_n$ & The set of all byte sequences of length $n$ bytes: $\mathbb{B}_n = \{ B: B \in \mathbb{B} \wedge \lVert B \rVert = n \}$ \\
$\mathbb{P}$ & The set of positive integers [what's wrong with $\mathbb{N}$??? Grrr...]. \\
$\mathbb{P}_n$ & The set of all positive integers smaller than $2^n$: $\mathbb{P}_n = \{ P: P \in \mathbb{P} \wedge P < 2^n \}$ \\
$M_{3:2048}$ & Specialised Bloom filter. \\
$\Lambda(...)$ & Contract creation function. \\
$\Theta(...)$ & ``Message call''/contract execution function? Not very clearly defined anywhere, but used extensively. \\
$\Gamma(B)$ & The ``initiation state'' of block $B$. Usually $\boldsymbol{\sigma}_i: \mathtt{TRIE}(L_S(\boldsymbol{\sigma}_i)) = {P(B_H)_H}_r$. \\
$\Psi(B)$ & A block transition function that maps an incomplete block $B$ to a complete block $B'$ (adds in mixHash, nonce, stateRoot). \\
$r(...)$ & Calculates stateRoot? Used once but not defined. \\
\textit{etc.} \\

\bottomrule
\end{longtable}

\end{document}
